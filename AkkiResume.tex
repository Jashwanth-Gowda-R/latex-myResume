\documentclass[11pt,a4paper,sans]{moderncv}
\moderncvstyle{banking}
\moderncvcolor{blue}
\usepackage[utf8]{inputenc}                    
\usepackage[scale=0.75]{geometry}
\usepackage{import}

\name{Akhilesh}{Kakolu Ramarao \break}
\email{kakolura@hhu.de}                        
\homepage{akkikek.xyz}

\begin{document}

\makecvtitle

\section{Experience}

\vspace{3pt}

\subsection{Employment}

\vspace{6pt}

\begin{itemize}

\item{\cventry{2021-Present}{Wissenschaftlicher Mitarbeiter, Department of English Language and Linguistics}{\href{https://www.hhu.de/}{Heinrich-Heine-Universität Düsseldorf
} }{Germany}{}{}}

\begin{itemize}
    \item Working on Computational Morphology. Supervised by \href{https://kevintang.org}{Prof. Dr. Kevin Tang} and mentored by \href{https://blogs.phil.hhu.de/dbh13/}{Dr. Dinah Baer-Henney}.
    
    \item Teaching Computational Linguistics courses for Bachelor of Arts students. 

    \item IT administration for \href{https://slam.phil.hhu.de/}{\underline{Slamlab}}.
    
\end{itemize}

\vspace{6pt}


\item{\cventry{2019 - 2020}{NLP Researcher}{\href{https://civicdatalab.in/}{CivicDataLab} }{Remote}{}{}}

\begin{itemize}
    \item Played an integral role in developing and maintaining partnerships with various Non-Governmental Organizations and Governments.
    
    \item Worked on building Language Platform for Assamese language that includes - Optical Character Recognition (for Assamese), Machine Translation system for English and Assamese, Assamese keyboard, Rich text editor and creation of Parallel Corpora for English and Assamese.
    
    \item Case studies: 
    
    \begin{itemize}
        \item Analysing language complexity of Union Budget Speeches.
        \item Topic modeling on Union Budget suggestions to understand citizens’ input for the Union Budget 2020.
    \end{itemize}
    
    \item Built Crowd-sourced Translation platform to aid the process of disseminating credible COVID-19 related information in local languages.
    
    \item Involved in building Text Annotation platform to perform Sequence Labeling. It was being used to annotate entities and relations in Indian Court judgements.

\end{itemize}

\vspace{6pt}

\item{\cventry{2017 - 2019}{NLP Engineer}{\href{https://vernacular.ai/}{Vernacular.ai}}{India}{Bengaluru}{}}
\begin{itemize}
    \item Early stage of the company presented me with an  opportunity to build many technical components involved in both Voice Assistants and Multi-lingual Chatbots from the ground up.
    \item Involved in designing and developing end-to-end speech and natural language processing A.I. systems for automating Customer Care Systems for Hotel industry.
    \item Adopted various strategies to enrich customer
    experience such as: conversation design framework, dialog management system, predicting user responses and so on.


\end{itemize}

\vspace{6pt}

\item{\cventry{2017}{Software Engineer}{Ramco Systems}{India}{Bengaluru/Chennai}{}}

\begin{itemize}
    \item Prime developer in Designing and Developing a Chatbot for Human Resource (HR) system.

    \item Responsible for all the Bug fixes, Documentation and feature implementation of the Machine Learning components.

    \item Crafted users’ conversations and implemented them using Finite State Machines.
    
    \item Mentions: Top 20 apps on Microsoft teams.

\end{itemize}

\vspace{6pt}

\end{itemize}



\subsection{Language Revitalization}

\item{\cventry{2020}{Chief Coordinator}{\href{https://imcls.org}{Idu Mishmi Cultural and Literary Society (IMCLS)}}{India}{Arunachal Pradesh}{}}
    
    \begin{itemize}
    
    \item Played an instrumental role in initiating the Language Revitalization Movement by managing Partnerships and Technology as a Chief Coordinator of IMCLS, the apex body of Idu Mishmi Tribe in Arunachal Pradesh, India.

    \item Built the first ever E-Dictionary for Idu Mishmi and English in the form of a \underline{\href{https://play.google.com/store/apps/details?id=imcls.dictionary.demo_flutter}{ Mobile Dictonary application}}.
    
    \item Coverage: 
    
    \begin{itemize}
        \item \underline{\href{https://arunachaltimes.in/index.php/2020/12/14/imcls-launches-mishmi-idu-dictionary-app/}{Launch of the E-dictionary.}} 
        
        \item \underline{\href{https://m.facebook.com/story.php?story_fbid=4876201042454353&id=1037093553031807&sfnsn=wiwspwa}{Promotion from the Chief Minister of Arunachal Pradesh.}} 
        
    \end{itemize}
\end{itemize}


\vspace{6pt}

\subsection{Consulting}

\begin{itemize}

\vspace{6pt}

\item{\cventry{2021}{Consultant}{XRI Institute}{Remote}{United States of America}{}}

\begin{itemize}

    \item Developed tools that serve the development of Human Language Technology (HLT) for low-resource languages.

\end{itemize}

\vspace{6pt}

\item{\cventry{2020 - 2021}{Tech Consultant}{\href{https://vrook.co}{Vrook}}{India}{}{}}

\begin{itemize}

    \item Involved in planning and implementing strategies for Online pedagogy.

    \item Developing technology plan and providing continuous support in building learner-centric online learning platform.

\end{itemize}

\end{itemize}

\vspace{6pt}

\subsection{Internships}

\begin{itemize}

\vspace{6pt}

\item{\cventry{2016}{Full Stack Developer Intern}{\href{http://sense8.tech}{Sense Infinity Technologies}}{India}{}{}}

\begin{itemize}
    \item Designed and Developed a Cross-Platform \underline{\href{https://staple.today/}{Food Delivery Application}} written with AngularJS (Ionic Framework and Cordova). 

    \item Involved in building backend for a \underline{\href{https://play.google.com/store/apps/details?id=com.sense.today.ballyhoo}{Restaurant Booking Application}} using PHP (Laravel).

\end{itemize}

\vspace{6pt}

\item{\cventry{2014}{Industry Trainee}{Maruti Suzuki}{India}{}{}}

\begin{itemize}

    \item Worked with Floor supervisors and Mechanics as an Industry trainee.

\end{itemize}

\end{itemize}

\vspace{6pt}

\subsection{Volunteering}

\begin{itemize}

\vspace{6pt}

\item{\cventry{2020 - Present}{Volunteer}{\href{http://www.lohit-libraries.org}{Lohit Youth Library Network}}{India}{Arunachal Pradesh}{\vspace{3pt} Founded by Padmashree Sathyanarayanan Mundayoor, Lohit  Youth  Library  Network is a  unique youth initiative in  Arunachal  Pradesh  in  North-eastern Himalayan  India, reaching  out  to readers  across  a  span  of  300  kms in  the  remote  Lohit  and Dibang Valley regions, offering free reading & learning opportunities.}}

\vspace{6pt}

\item{\cventry{2012-2020}{Core Committee Member}{\href{http://panchajanya.org/}{Panchajanya Foundation}}{India}{Bengaluru}{\vspace{3pt}Our work is aimed at enriching and enhancing the capabilities of individuals to achieve well-being and harmony with respect to ‘Akshara’ – Knowledge, ‘Arogya’ – Health and ‘Adhyatma’ – Spiritual and moral values.}}

\end{itemize}

\vspace{6pt}

\subsection{Freelancing}

\begin{itemize}

\vspace{6pt}

\item{Designed and Developed a search tool for Online (Business) Listings using Python and Elasticsearch in 2017.}

% \item {Built website for Pelican Biotech in 2016.}

\item {Reviewed code and designed a few technical components for a Stealth Startup in 2017.}

\end{itemize}

\section{Education}

\vspace{3pt}

\subsection{Academic Qualifications}

\vspace{6pt}

\begin{itemize}

% \item{\cventry{2021 - Present}{PhD Student}{Heinrich Heine University}{Düsseldorf, Germany}{\textit{Department of English language and Linguistics}}{}}

\item{\cventry{2012 - 2016}{Undergraduate}{Dayanand Sagar College of Engineering}{Bengaluru, India}{\textit{Bachelor of Automobile Engineering}}{}}

\end{itemize}

\vspace{6pt}

\section{Teaching}\label{teaching}

\begin{itemize}
    \item \textbf{Accent unplugged: Evaluating Voice assistants} \href{https://lsf.hhu.de/qisserver/rds?state=verpublish&status=init&vmfile=no&publishid=244439&moduleCall=webInfo&publishConfFile=webInfo&publishSubDir=veranstaltung}{\small{\underline{[Link]}}} \qquad \qquad \qquad \qquad \qquad \quad Winter, 2023
    \item \textbf{Synthesizing speech} \href{https://lsf.hhu.de/qisserver/rds?state=verpublish&status=init&vmfile=no&publishid=240679&moduleCall=webInfo&publishConfFile=webInfo&publishSubDir=veranstaltung}{\small{\underline{[Link]}}} \qquad \qquad \qquad \qquad \qquad \qquad \qquad \qquad \qquad \qquad \qquad Summer, 2023
    \item \textbf{Language technology for Linguists with Internet of Things (IoT)} \href{https://lsf.hhu.de/qisserver/rds?state=verpublish&status=init&vmfile=no&publishid=231180&moduleCall=webInfo&publishConfFile=webInfo&publishSubDir=veranstaltung}{\small{\underline{[Link]}}} \qquad Winter, 2022
    \item \textbf{LaTeX for Linguists} \href{https://lsf.hhu.de/qisserver/rds?state=verpublish&status=init&vmfile=no&publishid=235036&moduleCall=webInfo&publishConfFile=webInfo&publishSubDir=veranstaltung}{\small{\underline{[Link]}}} \qquad \qquad \qquad \qquad \qquad \qquad \qquad \qquad \qquad \qquad \qquad \quad Winter, 2022
    \item \textbf{Towards a career in language technology: linguistic annotation} \href{https://lsf.hhu.de/qisserver/rds?state=verpublish&status=init&vmfile=no&publishid=227683&moduleCall=webInfo&publishConfFile=webInfo&publishSubDir=veranstaltung}{\small{\underline{[Link]}}} \qquad \quad Summer, 2022
    \item \textbf{Basic Skills in Digital Humanities} \href{https://lsf.hhu.de/qisserver/rds?state=verpublish&status=init&vmfile=no&publishid=220184&moduleCall=webInfo&publishConfFile=webInfo&publishSubDir=veranstaltung&noDBAction=y&init=y}{\small{\underline{[Link]}}} \qquad \qquad \qquad \qquad \qquad \qquad \qquad  \qquad \quad Winter, 2021
    
\end{itemize}

\section{Skills and Projects}

\vspace{3pt}

\subsection{Technical Skills}

\begin{itemize}

\vspace{6pt}

\item \textbf{Recent experience:} Python, Bash, \LaTeX, Scheme  
\vspace{6pt}

\item \textbf{Past experience:} Haskell, Dart, JavaScript (AngularJS), \small{\textbf{Frameworks:}} Keras, Tensorflow, Flutter

\vspace{6pt}

\item \textbf{Platforms:} Arch linux, Ubuntu

\end{itemize}

\vspace{6pt}

\subsection{Personal Projects}

\vspace{6pt}

\begin{itemize}

\item My personal technical projects and open source contributions can be found \underline{\href{https://github.com/akki2825}{here}}.

\vspace{6pt}

\item My non-technical project in Material Science was to \textbf{Analyse Corrosion and Wear behaviour of developed Fly-ash based Thermal Spray coating.}

\end{itemize}

\vspace{6pt}

\section{Grants/Awards}

\vspace{3pt}

\begin{itemize}
    \item \textbf{First prize} at the \textbf{\underline{\href{https://www.heicad.hhu.de/aktivitaeten/hhu-legal-hackathon-2022}{HHU Legal Hackathon 2022}}}. Team: Akhilesh Kakolu Ramarao, Phaedon Paschalis and Diana Schill.
    
    \item \textbf{First prize} at the \textbf{\underline{\href{https://www.heicad.hhu.de/aktivitaeten/lightning-talks-2022}{HeiCAD Lightning Talks 2022}}}.
    
    \item Hardware Grant. 20 × NVIDIA® JetsonTM Nano Developer Kits. NVIDIA Academic Hardware Grant, “FOSStering Digital Humanities through Accent Diversity \& Conversational Devices”. Co-applicant: Kevin Tang
    
    \item Education Grant. €9,424. e-Learning support fund, Heinrich-HeineUniversity Düsseldorf. “Interactive web-based review units for phonetics and phonology”. Co-applicant: Christopher Geissler and Kevin Tang 
    
\end{itemize}

\section{Publications}

\vspace{3pt}

\begin{itemize}
  \item Chongnam Jeong, Dominic Schmitz, \textbf{Akhilesh Kakolu Ramarao}, Anna Stein, and Kevin Tang. 2023. Linear discriminative learning: a competitive non-neural baseline for morphological inflection. In \textit{Proceedings of the 20th SIGMORPHON Workshop on Computational Research in Phonetics, Phonology, and Morphology}, pages 138–150, Toronto, Canada. Association for Computational Linguistics.
  \url{https://aclanthology.org/2023.sigmorphon-1.16.pdf}
  \item \textbf{Akhilesh Kakolu Ramarao}, Yulia Zinova, Kevin Tang \& Ruben van de Vijver. 2022. HeiMorph at SIGMORPHON 2022 shared task on morphological acquisition trajectories. In \textit{Proceedings of the 19th SIGMORPHON Workshop on Computational Research in Phonetics, Phonology, and Morphology}, 236–239. Seattle, Washington: Association for Computational Linguistics. \url{https:// aclanthology.org/2022.sigmorphon-1.24} 
\end{itemize}


\section{Presentations}

\subsection{Talks}
    \begin{itemize}
        \item \textbf{'From Zero to Terminal Hero'} at 32nd \href{https://tacosconference.github.io/agenda}{\underline{TaCoS conference}}, 2023.
        \item \textbf{Akhilesh Kakolu Ramarao}, Kevin Tang \& Dinah Baer-Henney. January, 2023. Can Neural Networks learn human-like behavior when inflecting verbs? the case of Spanish. Shortlisted presentation at the Five-Minute linguist competition, \textit{Linguistic Society of America 97th Annual Meeting 2023}, Denver,
CO, USA.
        \item \textbf{'Modeling irregular morphological patterns using Transformers'} at  \underline{\href{https://www.heicad.hhu.de/aktivitaeten/lightning-talks-2022}{HeiCAD Lightning Talks 2022}}.
        \item \textbf{'Language Revitalization: A case for Idu Mishmi'} at Microsoft Research India, 2021.
        \item \textbf{'Language Revitalization: A case for Idu Mishmi'} at \href{https://www.navana.ai/}{Navana Tech}, 2021.
    \end{itemize}

\subsection{Poster presentations}
    \begin{itemize}
        \item (June 2023) Muschalik, J., Schmitz, D., \textbf{Kakolu Ramarao, A.} \& D. Baer-Henney. Acoustic duration and typing timing – same, same… but different? Presentation at Phonetics and Phonology in Europe 2023, Nijmegen, Netherlands.
        \item Kevin Tang, \textbf{Akhilesh Kakolu Ramarao} \& Dinah Baer-Henney. May, 2022. Modeling irregular morphological patterns with recurrent neural network: the case of the L-shaped morphome. \textit{13th Mediterranean Morphology Meeting}, University of the Aegean, Greece.
        \item \textbf{Akhilesh Kakolu Ramarao. 2020}. Analysis of POCSO cases in Uttar Pradesh. Agami.
    \end{itemize}

\section{Conference/Workshop Organiser}
\begin{itemize}
    \item Chief organizer of LibreSlam - Free and Open Source (FOSS) for everybody (\url{https://slam.phil.hhu.de/libreslam/}) at Heinrich-Heine-University, Germany, 2022.
    \item Chief organiser and a teaching member of the \underline{{\href{https://youtube.com/playlist?list=PLzgT9LNklpYDFav5-jj4IiRYVP91ufMtf}{LY-CS101 Course}}}. The course was primarily aimed at the beginner level Computer Science students in collaboration with National College, Bangalore, India and \href{https://www.mergeintern.com/}{Merge Intern}, 2020.
\end{itemize}

\vspace{8pt}

References available upon request.
% \section{References}

% \vspace{6pt}
 
% \begin{itemize}

% \item{Up to 3 references available on request}

% \end{itemize}

% Publications from a BibTeX file without multibib
%  for numerical labels: \renewcommand{\bibliographyitemlabel}{\@biblabel{\arabic{enumiv}}}% CONSIDER MERGING WITH PREAMBLE PART
%  to redefine the heading string ("Publications"): \renewcommand{\refname}{Articles}
\nocite{*}
\bibliographystyle{plain}
\bibliography{publications}                    

\end{document}